\setlength{\parindent}{1em}
\title{IMP++}
\author{Grigore Ro\c{s}u (\texttt{grosu@illinois.edu})}
\organization{University of Illinois at Urbana-Champaign}

\maketitle

\begin{latexComment}
\section{Abstract}
This is the \K semantic definition of the IMP++ language.
IMP++ extends the IMP language with the features listed below.  We
strongly recommend the reader to familiarize with IMP and its \K
definition before proceeding.
\begin{itemize}
\item Strings and concatenation of strings.  Strings are useful
for the \texttt{print} statement, which is discussed below.  For
string concatenation, we use the same \texttt{+} construct that we use
for addition.
\item Variable increment.  We only add a preincrement construct:
\texttt{++x} increments variable \texttt{x} and evaluates to the
incremented value.  Variable increment makes the evaluation of
expressions to have side effects, and thus the evaluation strategies
of the various language constructs to matter in what regards the set
of possible program behaviors.
\item Input and output.  IMP++ adds a \texttt{read()} expression
construct which reads an integer number and evaluates to it, and 
a variadic (i.e., it has an arbitrary number of arguments) statement
construct \texttt{print(e1,e2,...,en)} which evaluates its arguments
and then outputs their values.  Note that the \K tool allows to
connect the input and output cells to the standard input and output
buffers, this way compiling the language definition into an
interactive interpreter.
\item Abrupt termination.  The \texttt{halt} statement simply halts
the program.  The \K tool shows the resulting configuration, as if the
program terminated normally.
\item Dynamic threads. The statement \texttt{spawn s} starts a new
concurrent thread that executes statement \texttt{s}.  The newly
created thread is given at creation time the {\em environment} of its
parent, so it can access all its parent's variables.  This allows for
the parent thread and the child thread to communicate; it also allows
for races and ``unexpected'' behaviors, so be careful.
\item Blocks and local variables.  IMP++ allows blocks enclosed by 
curly brackets.  Also, IMP's global variable declaration construct is
generalized to be used anywhere as a statement, not only at the
begining of the program.  As expected, the scope of the declared
variables is from their declaration point till the end of the most
nested enclosing block.
\end{itemize}

\end{latexComment}

\vspace*{3ex}
