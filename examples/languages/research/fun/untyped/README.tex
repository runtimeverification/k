\setlength{\parindent}{1em}
\title{FUN --- Untyped}
\author{Grigore Ro\c{s}u and 
        Traian Florin \c{S}erb\u{a}nu\c{t}\u{a} (\texttt{\{grosu,tserban2\}@illinois.edu})}
\organization{University of Illinois at Urbana-Champaign}

\maketitle

\begin{latexComment}
\section{Abstract}
This is the \K semantic definition of the untyped FUN language.
FUN is intended to be a pedagogical and research language that captures
the essence of the functional programming paradigm, extended with several
features often encountered in functional programming languages.
Like many functional languages, FUN is an expression language,
that is, everything, including the main program, is an expression.
Functions can be declared anywhere and are first class values in the
language.  FUN is call-by-value here, but it has been extended (as
student homework assignments) with other parameter-passing styles.
To make it more interesting and to highlight some of \K's strengths,
FUN includes the following features in addition to the conventional
functional constructs encountered in similar languages used as
teaching material:
\begin{itemize}
\item Functions can take multiple arguments in two different ways.
First, they can take space-separated arguments whose semantics is given by
currying.  Second, they can take comma-separated tuple arguments, whose
semantics is given directly, not via currying.  For example, FUN allows
function declarations/invocations of the form
``\texttt{f (a,b) c  (d,e)}''.
\item Similarly, we allow \texttt{let} and \texttt{letrec} binders
which work with lists of variables and expressions, and we give
their semantics directly, without desugaring them to one-argument variants.
We also allow the usual syntactic sugar for declaring-and-binding functions
with ``\texttt{let f (a,b) c (d,e) = ...}''.
\item We include a \texttt{callcc} construct, for two reasons: first,
several functional languages support this construct; second, some
semantic frameworks have a hard time defining it.
\item Finally, we include mutables by means of referencing and
expression, getting the reference of a variable, dereferencing and
assignments.  We include these for the same reasons as above: there
are languages which have them, and they are not easy to define in some
semantic frameworks.
\end{itemize}
Like in many other languages, some of FUN's constructs can be
desugared into a smaller set of basic constructs.  We do that in a 
dedicated module between the syntax and the semantics, and we only
give semantics to the core constructs.

\paragraph{Note:}{
We recommend the reader to first consult the dynamic semantics (at
least the environment-based definition) of the EXP language (under
examples/language/classic/exp).  To keep the comments below small and
focused, we will not re-explain functional or \K features that have
already been explained in the definition of EXP.
}

\end{latexComment}

\vspace*{3ex}
