% Type the command ``make pdf'' to generate the PDF poster of this language.
% Alternatively, type in the equivalent command
% kompile.pl simple-untyped.k -l SIMPLE-UNTYPED -style bb -pdf SIMPLE-UNTYPED-SYNTAX SIMPLE-UNTYPED-DESUGARED-SYNTAX SIMPLE-UNTYPED-SEMANTICS SIMPLE-UNTYPED -topmatter README.tex
% The generated simple-untyped.pdf file is easier to read than the actual ASCII files.

\setlength{\parindent}{1em}
\title{FUN --- Untyped}
\author{Grigore Ro\c{s}u and 
        Traian Florin \c{S}erb\u{a}nu\c{t}\u{a} (\texttt{\{grosu,tserban2\}@illinois.edu})}
\organization{University of Illinois at Urbana-Champaign}

\maketitle

\begin{latexComment}
\section{Abstract}
This is the \K semantic definition of the untyped FUN language.
FUN is intended to be a pedagogical and research language that captures
the essence of the functional programming paradigm, extended with several
features often encountered in functional programming languages.
Like many functional languages, FUN is an expression language,
that is, everything, including the main program, is an expression.
Functions can be declared anywhere and evaluate to closures, which
are first class values in the language.  To make it more
interesting and to highlight some of \K's strengths, FUN includes
the following features in addition to the conventional functional
constructs encountered in similar languages used as teaching material:
\begin{itemize}
\item Functions can take multiple arguments in two different ways.
First, they can take space-separated arguments whose semantics is given by
currying.  Second, they can take comma-separated tuple arguments, whose
semantics is given directly, not via currying.  For example, FUN allows
function declarations/invocations of the form
``\texttt{f (a,b) c  (d,e)}''.
\item Similarly, we allow \texttt{let} and \texttt{letrec} binders
which work with lists of variables and expressions, and we give
their semantics directly, without desugaring them to one-argument variants.
We also allow the usual syntactic sugar for declaring-and-binding functions
with ``\texttt{let f (a,b) c (d,e) = ...}''.
\item We include a \texttt{callcc} construct, for two reasons: first,
several functional languages support this construct; second, some
semantic frameworks have a hard time defining it.
\item Finally, we include mutables by means of referencing, dereferencing and
assignments.  We include these for the same reasons as above: there are
functional languages which have them, and they are not easy to define
in some semantic frameworks.
\end{itemize}
Like in many other languages, some of FUN's constructs can be
desugared into a smaller set of basic constructs.  We do that in a 
dedicated module between the syntax and the semantics, and we only
give semantics to the core constructs.

\paragraph{Note:}{
For a quick introduction to the \K prototype, you are referred to
the README file at the root of this k-framework distribution.  If you are
interested in reading more about \K, please check the following paper:
\begin{quote}
Grigore Ro\c su, Traian-Florin \c Serb\u anu\c t\u a:
\href{http://dx.doi.org/10.1016/j.jlap.2010.03.012}{An overview of the K semantic framework}.

Journal of Logic and Algebraic Programming, 79(6): 397-434 (2010)
\end{quote}
}
\end{latexComment}

\vspace*{3ex}
